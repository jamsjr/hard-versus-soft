\documentclass{article}
\usepackage{amsmath}
\usepackage{graphicx}
\usepackage{url}
\usepackage[utf8x]{inputenc}
\usepackage[T1]{fontenc}
%\usepackage[brazil]{babel}
\usepackage{color}
\usepackage{times}
\usepackage{kchicago}

\newcommand{\comentario}[1]{}


\author{George Lima, Alexandre Passos, José Augusto Matos Santos}
\title{Hard Reservation vs Soft Reservation for soft real-time systems}

\begin{document}
\maketitle


\section{Introduction}
\label{sec:introduction}


\comentario{ Grande quantidade de trabalhos relativos ao uso de
  reservação de banda. Falta de estudos comparativos sobre
  hard/soft. Conexão com aplicações que poderiam se beneficiar do
  estudo.

  ** Adaptatividade
   
  falar que imaginam que hard é mais adaptativo que soft; mentira,
  dizer que a simulação desprova isso 

* Trabalhos relacionados
  
  Procurar
}

\section{Metrics}
\label{sec:metrics}

\comentario{   
* Métricas comparativas
  
  Descrição e conexão com aplicações.

  
** Tempo de resposta
   
   
** Intervalo entre deadlines
}

\section{Simulation environment}
\label{sec:simul-envir}

\comentario{
* Simulação
  
  Descrição do ambiente, modelo de tarefas, dados de entrada, etc.
}

\section{Simulation Results}
\label{sec:simulation-results}

\section{Conclusion}
\label{sec:conclusion}



\bibliographystyle{kchicago}
\bibliography{bib}
\end{document}
