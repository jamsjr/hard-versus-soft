\documentclass{article} % really? Olhar qual a conferência, incorporar
                        % logo estilo e bibestilo
\usepackage{amsmath}
\usepackage{graphicx}
\usepackage{url}
\usepackage[utf8x]{inputenc}
\usepackage[T1]{fontenc}
%\usepackage[brazil]{babel}
\usepackage{color}
\usepackage{times}
\usepackage{kchicago}

\newcommand{\comentario}[1]{}
\newcommand{\superscript}[1]{\ensuremath{^{\textrm{#1}}}}
\newcounter{notecounter}
\newcommand{\nota}[1]{\addtocounter{notecounter}{1}\superscript{\textcolor{red}{\arabic{notecounter}}}\marginpar{\textbf{\arabic{notecounter}:
    }#1}}

\newcommand{\simul}{\textbf{Simul}} % Que criatividade. Mudar esse
% nome depois, urgente

\author{George Lima, Alexandre Passos, José Augusto Matos Santos}
\title{Hard Reservation vs Soft Reservation for soft real-time systems}

\begin{document}
\maketitle

\begin{abstract}

In reservation-based scheduling of soft real-time tasks there are two
broad approaches to making use of the reservation allotted to each
server. In a \textbf{hard} reservation framework each server is
allowed a given fraction of the total processing time, and in no
ocasion can use more than that fraction. If in some circumstances, a
pending task that has overrun its budget is allowed to use idle
processing time, then the reservation is said to be
\textbf{soft}\nota{colocar referência da definição}. It is a truth
widely acknowledged that a system based on hard reservation is more
predictable, if not more efficient, than a system based on soft
reservation. In this paper we present a simulation environment for
becnhmarking both reservation mechanisms and experimental results in
this environment that fail to show any significant benefit either in
predictability or in performance.
  
\end{abstract}

\section{Introduction}
\label{sec:introduction}



\comentario{ Grande quantidade de trabalhos relativos ao uso de
  reservação de banda. Falta de estudos comparativos sobre
  hard/soft. Conexão com aplicações que poderiam se beneficiar do
  estudo.

  ** Adaptatividade
   
  falar que imaginam que hard é mais adaptativo que soft; mentira,
  dizer que a simulação desprova isso 

* Trabalhos relacionados
  
  Procurar
}

\section{Simulation environment}
\label{sec:simul-envir}

To perform the experiments presented in this paper we implemented
\simul{}, a simple python-based simulator for real-time scheduling
algorithms. Its source code, the data files used in this paper and
instructions to reproduce our results are available in
\url{http://github.com/jamsjr/hard-versus-soft/tree/master}. \simul{}
implements a basic EDF scheduler and, on top of it, runs both hard
real-time periodic tasks and bandwidth sharing servers. These servers
can be either traditional CBS servers \nota{colocar referência para
  CBS} or a hard reservation adaptation of the CBS algorithm. Soft
real-time tasks running on these servers either have their execution
times sampled from a probability distribution or follow traces
generated from a modified version of mplayer\nota{citar mplayer?} that
reports the start time and decoding cost for each frame in a
high-definition video running on a XXX\nota{pegar info do pc de
  eduardo}.

\subsection{CBS}
\label{sec:cbs}



\comentario{
* Simulação
  
  Descrição do ambiente, modelo de tarefas, dados de entrada, etc.
}


\section{Performance Metrics}
\label{sec:metrics}

\comentario{   
* Métricas comparativas
  
  Descrição e conexão com aplicações.

  
** Tempo de resposta
   
   
** Intervalo entre deadlines
}


\section{Simulation Results}
\label{sec:simulation-results}

\section{Conclusion}
\label{sec:conclusion}



\bibliographystyle{kchicago}
\bibliography{bib}
\end{document}
